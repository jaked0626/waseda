% Options for packages loaded elsewhere
\PassOptionsToPackage{unicode}{hyperref}
\PassOptionsToPackage{hyphens}{url}
%
\documentclass[
]{ltjarticle}
\usepackage{lmodern}
\usepackage{amssymb,amsmath}
\usepackage{ifxetex,ifluatex}
\ifnum 0\ifxetex 1\fi\ifluatex 1\fi=0 % if pdftex
  \usepackage[T1]{fontenc}
  \usepackage[utf8]{inputenc}
  \usepackage{textcomp} % provide euro and other symbols
\else % if luatex or xetex
  \usepackage{unicode-math}
  \defaultfontfeatures{Scale=MatchLowercase}
  \defaultfontfeatures[\rmfamily]{Ligatures=TeX,Scale=1}
\fi
% Use upquote if available, for straight quotes in verbatim environments
\IfFileExists{upquote.sty}{\usepackage{upquote}}{}
\IfFileExists{microtype.sty}{% use microtype if available
  \usepackage[]{microtype}
  \UseMicrotypeSet[protrusion]{basicmath} % disable protrusion for tt fonts
}{}
\makeatletter
\@ifundefined{KOMAClassName}{% if non-KOMA class
  \IfFileExists{parskip.sty}{%
    \usepackage{parskip}
  }{% else
    \setlength{\parindent}{0pt}
    \setlength{\parskip}{6pt plus 2pt minus 1pt}}
}{% if KOMA class
  \KOMAoptions{parskip=half}}
\makeatother
\usepackage{xcolor}
\IfFileExists{xurl.sty}{\usepackage{xurl}}{} % add URL line breaks if available
\IfFileExists{bookmark.sty}{\usepackage{bookmark}}{\usepackage{hyperref}}
\hypersetup{
  pdftitle={Economic Policy - Notes},
  pdfauthor={Jake Underland},
  hidelinks,
  pdfcreator={LaTeX via pandoc}}
\urlstyle{same} % disable monospaced font for URLs
\usepackage[margin=3cm]{geometry}
\usepackage{graphicx,grffile}
\makeatletter
\def\maxwidth{\ifdim\Gin@nat@width>\linewidth\linewidth\else\Gin@nat@width\fi}
\def\maxheight{\ifdim\Gin@nat@height>\textheight\textheight\else\Gin@nat@height\fi}
\makeatother
% Scale images if necessary, so that they will not overflow the page
% margins by default, and it is still possible to overwrite the defaults
% using explicit options in \includegraphics[width, height, ...]{}
\setkeys{Gin}{width=\maxwidth,height=\maxheight,keepaspectratio}
% Set default figure placement to htbp
\makeatletter
\def\fps@figure{htbp}
\makeatother
\setlength{\emergencystretch}{3em} % prevent overfull lines
\providecommand{\tightlist}{%
  \setlength{\itemsep}{0pt}\setlength{\parskip}{0pt}}
\setcounter{secnumdepth}{-\maxdimen} % remove section numbering
\usepackage{amsmath}
\usepackage{color}
\usepackage{hyperref}
\usepackage{luatexja}

\title{Economic Policy - Notes}
\usepackage{etoolbox}
\makeatletter
\providecommand{\subtitle}[1]{% add subtitle to \maketitle
  \apptocmd{\@title}{\par {\large #1 \par}}{}{}
}
\makeatother
\subtitle{Economic Policy Fall, 2021}
\author{Jake Underland}
\date{2021-10-30}

\begin{document}
\maketitle

{
\setcounter{tocdepth}{3}
\tableofcontents
}
\hypertarget{intro}{%
\section{Intro}\label{intro}}

\hypertarget{functions}{%
\subsection{Functions}\label{functions}}

\begin{itemize}
\item
  Functions and curves are relationships that map a field to another
  field. \[f:\underbrace{A}_\textit{domain} \to B\]
\item
  \emph{Demand curve} maps price to demand, \emph{cost curve} maps price
  to cost.
\item
  In most graphs, the domain is taken on the x-axis and the image of A
  under the curve f is taken on the y-axis. However, when graphing
  supply and demand, custom dictates we do the opposite (price, which is
  the input, is on the y-axis and demand/supply is on the x-axis).
\end{itemize}

\hypertarget{partial-and-general-equilibrium-models}{%
\subsection{Partial and General Equilibrium
Models}\label{partial-and-general-equilibrium-models}}

\begin{itemize}
\tightlist
\item
  General Equilibrium models employ indifference curves. Partial
  equilibrium models employ cost and demand curves.
\end{itemize}

\hypertarget{goals-of-government}{%
\subsection{Goals of Government}\label{goals-of-government}}

\begin{itemize}
\item
  \emph{Efficiency} and \emph{equity}.
\item
  When focusing on efficiency, partial equilibrium models are
  sufficient.
\item
  In this class, we assume the government is only concerned with
  efficiency.
\item
  \emph{Efficiency} = Pareto efficiency -\textgreater{} there is no
  alternative allocation where all parties are left better off than they
  were before. Efficiency implies the maximization of net surplus (if
  net surplus can be further increased, then by increasing and adjusting
  the allocation ex-post, all parties can improve their share).
\item
  Consumer surplus = benefit - expenditure\\
  Utility is a more broadly defined concept and its conversion to
  quantitative price need not be defined.
\item
  Profit = Revenue - Cost
\end{itemize}

\hypertarget{normative-arguments-in-economics}{%
\subsection{Normative Arguments in
Economics}\label{normative-arguments-in-economics}}

\begin{itemize}
\tightlist
\item
  the value of a commodity is not decided extraneously but decided by
  the consumer subjectively (allows for freedom of individual values).\\
\item
  Society should respect that and policy should be aimed at maximizing
  that (but how do we measure an inherently subjective value judgment in
  an objective manner? \(\downarrow\))\\
\item
  Economics posits that if consumers are rational, such value judgments
  can be measured objectively by using prices to rank preferences.
  Further, it can be maximized.
\end{itemize}

\hypertarget{marginal-benefit-curve-and-demand-curve}{%
\subsection{Marginal Benefit Curve and Demand
Curve}\label{marginal-benefit-curve-and-demand-curve}}

\begin{itemize}
\item
  Marginal Benefit Curve = Demand Curve
\item
  However, Demand Curve
  \(= f: \textcolor{red}{price} \to \textcolor{blue}{consumption}\) and
  MBC
  \(= f:\textcolor{blue}{consumption} \to \textcolor{magenta}{marginal \: benefit}\)
\item
  How are they the same? -\textgreater{} if a consumer is rational, they
  will consume a product until their marginal benefit from the product
  is equal to its price (cost).
\item
  We build the demand curve by taking the rational consumption amount
  for each price (so the amount where marginal utility = price). The
  Demand Curve is just a plot of the marginal utility at each quantity
  of the commodity.
\item
  If the Demand Curve goes left to right, starting with an input price
  and outputting a level of demand, then the Marginal Benefit Curve goes
  right to left, starting with a level of consumption and outputting a
  marginal benefit (=represented as price).\\
\item
  Marginal Cost Curve and Supply Curve have exact same relationship.\\
\item
  \hypertarget{foo}{\textbf{Lies on a few assumptions:}}
\end{itemize}

\begin{enumerate}
\def\labelenumi{\arabic{enumi}.}
\tightlist
\item
  Consumers and producers are rational actors\\
\item
  Consumers and producers are price-takers.\\
\item
  (Simplification assumption but not necessary:) MB and MC are gradually
  decreasing/increasing.
\end{enumerate}

\hypertarget{analysis-of-net-surplus}{%
\subsection{Analysis of Net Surplus}\label{analysis-of-net-surplus}}

\begin{itemize}
\tightlist
\item
  Demand and supply curves give us surplus -\textgreater{} Extract
  normative information from positive information.
\end{itemize}

\hypertarget{deductive-positivistic-analysis}{%
\subsection{Deductive Positivistic
Analysis}\label{deductive-positivistic-analysis}}

\begin{itemize}
\item
  Policy \(\to\) Fundamentals (cost, benefit) \(\to\) desirable outcome
  (price, production amount)
\item
  Economics is able to take into consideration the channels of change.
\end{itemize}

\hypertarget{meaning-of-efficiency}{%
\section{Meaning of Efficiency}\label{meaning-of-efficiency}}

\begin{itemize}
\item
  What is net surplus? -\textgreater{} Net surplus can be defined in the
  absence of a market (i.e.~without supply and demand curves). Such a
  definition is necessary when evaluating and comparing the efficiency
  of different methods to distribute resources.
\item
  The net surplus of a given allocation method:
  \[\text{Total Benefit of all consumers} - \text{Total cost of all firms}\]
\item
  (Deriving the market equation for net surplus from the above:)
  \[\begin{aligned} 
  \text{Net Surplus} =& \text{Total Benefit of all consumers} - \text{Total cost of all firms} \\
  =& \textcolor{red}{(\text{Total Benefit of consumers} - \underbrace{\text{Total expenditure of consumers}}_\text{=total profit firms})}\\
  &+ \textcolor{blue}{(\underbrace{\text{Total Profit of firms}}_\text{= total expenditure consumers} - \text{Total cost of firms})} \\
  =& \textcolor{red}{\text{Consumer Surplus}} + \textcolor{blue}{\text{Producer Surplus}}
  \end{aligned}\]
\item
  \textbf{Definition of an efficient allocation}:\\
  When the benefit function (demand function) and each firm's cost
  function are defined, the feasible method of allocation which
  maximizes net surplus under these conditions is efficient.
\item
  (How do we know market is efficient? -\textgreater{} first welfare
  theorem)
\item
  \emph{Important fact:} If, under a given allocation, there exists at
  least one pair of consumer and firm such that the marginal benefit of
  the consumer and the marginal cost of the firm is not equivalent, then
  net surplus is not maximized (firm can produce more/less and consumer
  can consume more/less until their marginal costs and benefits are
  equivalent).\\
  Proof. \[\begin{aligned} 
  \text{If } D'(y) > C'(y):& \\
  & D(y+1) - D(y) > C(y+1) - C(y) \\
  &\implies \text{Total Surplus increases by } \Delta D(y) - \Delta C(y) \text{ when } y \to y + 1\\
  \text{If } D'(y) < C'(y):& \\
  & D(y) - D(y-1) < C(y) - C(y-1) \\
  &\implies \text{Total Surplus increases by } \Delta C(y) - \Delta 
  D(y) \text{ when } y \to y - 1
  \end{aligned}\]
\item
  By taking the contraposition to the above claim, we find that if an
  allocation is to be efficient, then \(\forall i, j:\) \[MU_i = MC_j\]
\item
  This condition is not dependent on the existence of a market or on the
  rationality of its actors. It can be applied \textbf{to any method of
  distribution}.
\item
  The efficiency of markets (under certain assumptions stated
  \hyperlink{foo}{here}) can be derived thus:\\
\end{itemize}

\begin{enumerate}
\def\labelenumi{\arabic{enumi}.}
\tightlist
\item
  In a competitive market, \(p = MU\) and
  \(p = MC\)\footnote{Costs factored into this equalization include opportunity cost}.\\
\item
  All consumers/producers face the same prices (\emph{Law of one
  price})\\
\item
  \(\implies\) marginal benefit and marginal cost between all consumers
  and firms are equal.
\end{enumerate}

\hypertarget{imperfect-competition}{%
\section{Imperfect Competition}\label{imperfect-competition}}

\hypertarget{the-inefficiencies-of-monopolies}{%
\subsection{The inefficiencies of
monopolies}\label{the-inefficiencies-of-monopolies}}

\begin{itemize}
\item
  Monopolists cease production when \(MR = p\), and since they are not
  price takers, \(MR > MC\) (they can control price, endogenizing price
  and demand into the marginal revenue maximization problem).\\
\item
  Since consumers still consume to the point where \(MU = p\), as a
  result \(MU > MC\), which defies the conditions for efficiency:
  \(p = MU = MC\).\\
\item
  The inefficiencies of monopolies, duopolies, oligopolies comes from
  insufficient production.\\
\item
  Since taxes only further decrease production, how do we restore
  efficiency in the case of imperfect competition?\\
\item
  \text{独占禁止法:「優越的地位の濫用」}
\item
  Types of monopolies:

  \begin{itemize}
  \tightlist
  \item
    Firm taking advantage of a newly emerged market that other firms
    have not entered yet. These monopolies will resolve over time and
    the lucrative initial period inspires innovative incentive (to
    discover new markets).\\
  \item
    \textbf{Natural monopoly}: a monopoly in an industry in which high
    infrastructural costs and other barriers to entry relative to the
    size of the market give the largest supplier in an industry, often
    the first supplier in a market, an overwhelming advantage over
    potential competitors.
  \end{itemize}
\end{itemize}

\hypertarget{natural-monopolies}{%
\subsection{Natural monopolies}\label{natural-monopolies}}

\begin{itemize}
\tightlist
\item
  \textbf{Economy of Scale}:

  \begin{itemize}
  \tightlist
  \item
    Economies with increasing returns to scale (marginally decreasing
    Average Total Cost).\\
  \end{itemize}
\item
  Natural monopolies are typically borne of Fixed Costs, and is scaled
  commensurately with the size of the fixed cost relative to variable
  cost (initial infrastructure investment as seen in electricity, mobile
  carriers, railroads, television, pharmaceutical, etc).\\
\item
  Newcomers cannot make up for their initial investment.\\
\item
  \textbf{Counteractive Policies:}

  \begin{itemize}
  \tightlist
  \item
    Price regulations by the government (e.g.~railroad fees, highway
    fees).\\
  \item
    Breaking up the economy of scale into subgroups and deregulating the
    smaller subgroups less scaled in nature
    \footnote{smaller fixed cost relative to variable cost} (e.g.~the
    electricity industry, keeping regulations on transmission of
    electricity but deregulating generation and distribution).\\
    In the west, it is typical to separate railroad enterprises into
    rental of railroad infrastructure (regulated) and train
    cars/transportation and management of cars (deregulated).\\
    In Japan, the construction of highways is a government enterprise
    while management and administration is private (the right to which
    is auctioned off).\\
  \end{itemize}
\item
  \textbf{Network Externalities:}

  \begin{itemize}
  \tightlist
  \item
    the phenomenon by which the value or utility a user derives from a
    good or service depends on the number of users of compatible
    products. Network effects are typically positive, resulting in a
    given user deriving more value from a product as other users join
    the same network (e.g.~phones, SNS, OS, Office Applications, digital
    currency).\\
  \item
    Network externalities do not necessarily imply an efficiency problem
    (imperfect competition\footnote{i.e. the distortion of prices}) but
    they can be if the dominating firm distorts their prices.
  \end{itemize}
\end{itemize}

\hypertarget{excise-tax-subsidies}{%
\subsection{Excise Tax, Subsidies}\label{excise-tax-subsidies}}

\begin{itemize}
\tightlist
\item
  Price Elasticity of Demand:

  \begin{itemize}
  \tightlist
  \item
    sensitivity of demand to price fluctuations\\
  \item
    the shape of the curve of the demand function.
  \end{itemize}
\end{itemize}

\hypertarget{possible-questions}{%
\section{Possible Questions}\label{possible-questions}}

\end{document}
