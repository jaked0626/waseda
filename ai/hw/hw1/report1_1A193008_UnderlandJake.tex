\documentclass[a4paper,11pt]{jsarticle}

% Preamble
\usepackage{amsmath}
\usepackage{amsthm}


\begin{document}
\hfill 人工知能A \\
\hfill アンダーランド ジェイク \\
\hfill 1A193008 \\
\hfill \today

\bigskip
\begin{center}\textbf{レポート1}\end{center} 

\noindent \textbf{課題1} \\
「機械が考える」を実現するためには、人間を模倣すべきかあるいはその必要はないか?各自の考えを書いてください。\\~\\
機械学習や人工知能は表象同士を関連づける。例えば、ピザの画像データを「ピザ」というテキストデータに結びつけて識別する作用は、「考えている」とは言えないと思われる。機械が考えるにはシンボルグラウンディング問題を解決する、つまり概念と実世界対象をシームレスに行き来し、実世界対象を概念に、概念を実世界対象の一例に変換する容量(とそれを成すための抽象性、あるいは記号の指示対象を自身で指定する能力)が必要となる。人間が少ないデータから精度の高い外挿を行えるのは、多様に存在する具体的なデータに共通の抽象的な指示対象を見出し、それに心的世界内で記号を当てがい、尚且つその心的世界の記号を実在世界に接地することができるからである。この過程こそが「考える」ことであり、機械が行う表象同士の関連づけは単なる最適化である。私が思うには、機械がこのシンボルグラウンディング問題を解決する、あるいは記号とその指示対象の関連づけを自身で行えるようになるには、必然的に人間の持つような心的作用を組み込むことになり、人間を模倣する形で発展しなくてはならない。

\bigskip

\noindent \textbf{課題2} \\
最近、囲碁のプログラムが強くなりましたが、これには深層学習という手法を使っています。これには非常に多くの正解データと不正解のデータを与えて学習する必要があります。したがって、その学習結果は、全く新しい場面では効果がないかもしれません。\\
このような条件を考えても、知的であるとか知性があると言っていいと思うか。自らの考えを書きなさい。\\~\\
先の問題で述べた通り、少ないデータからの外挿が高い精度を持つには、シンボルグラウンディング問題が解決されている必要がある。それは、最近の機械やプログラムのようなアルゴリズムが行う表象同士の関連づけが膨大なデータを必要とする所以であり、この最適化の工程は「考える」ことと、その結果心的世界で形成される記号とその指示対象の体系(=「知性」)とは明確に区別されるものであると思われる。先の問題の話の延長になるが、カントによると、人間は実在世界を感知する感性、感性で得られた表象データを綜合して類型化し、概念として保持する(記号を当てがう)悟性、悟性で得られた経験的概念を普遍的原理に則って統合する理性を以てして初めて認識を持つと言える。知性を認識の賜物と考えるとすると、AIが知性を保つためには少なくとも感性を担うセンサーモーターシステム、悟性を担う高度なパターン認識(統計的AI)とそれらを概念として保持する仕組み、また普遍的原理によって経験的概念に基づいた記号的な推論を行う仕組みが必要であり、それは少なくとも統計的AIの領域に収まりきるような深層学習アルゴリズムにはないと考えます。

\bigskip

\end{document}