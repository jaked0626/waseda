% Options for packages loaded elsewhere
\PassOptionsToPackage{unicode}{hyperref}
\PassOptionsToPackage{hyphens}{url}
%
\documentclass[
]{article}
\usepackage{lmodern}
\usepackage{amssymb,amsmath}
\usepackage{ifxetex,ifluatex}
\ifnum 0\ifxetex 1\fi\ifluatex 1\fi=0 % if pdftex
  \usepackage[T1]{fontenc}
  \usepackage[utf8]{inputenc}
  \usepackage{textcomp} % provide euro and other symbols
\else % if luatex or xetex
  \usepackage{unicode-math}
  \defaultfontfeatures{Scale=MatchLowercase}
  \defaultfontfeatures[\rmfamily]{Ligatures=TeX,Scale=1}
\fi
% Use upquote if available, for straight quotes in verbatim environments
\IfFileExists{upquote.sty}{\usepackage{upquote}}{}
\IfFileExists{microtype.sty}{% use microtype if available
  \usepackage[]{microtype}
  \UseMicrotypeSet[protrusion]{basicmath} % disable protrusion for tt fonts
}{}
\makeatletter
\@ifundefined{KOMAClassName}{% if non-KOMA class
  \IfFileExists{parskip.sty}{%
    \usepackage{parskip}
  }{% else
    \setlength{\parindent}{0pt}
    \setlength{\parskip}{6pt plus 2pt minus 1pt}}
}{% if KOMA class
  \KOMAoptions{parskip=half}}
\makeatother
\usepackage{xcolor}
\IfFileExists{xurl.sty}{\usepackage{xurl}}{} % add URL line breaks if available
\IfFileExists{bookmark.sty}{\usepackage{bookmark}}{\usepackage{hyperref}}
\hypersetup{
  pdftitle={Finite Population Correction},
  pdfauthor={Jake Underland},
  hidelinks,
  pdfcreator={LaTeX via pandoc}}
\urlstyle{same} % disable monospaced font for URLs
\usepackage[margin=1in]{geometry}
\usepackage{graphicx,grffile}
\makeatletter
\def\maxwidth{\ifdim\Gin@nat@width>\linewidth\linewidth\else\Gin@nat@width\fi}
\def\maxheight{\ifdim\Gin@nat@height>\textheight\textheight\else\Gin@nat@height\fi}
\makeatother
% Scale images if necessary, so that they will not overflow the page
% margins by default, and it is still possible to overwrite the defaults
% using explicit options in \includegraphics[width, height, ...]{}
\setkeys{Gin}{width=\maxwidth,height=\maxheight,keepaspectratio}
% Set default figure placement to htbp
\makeatletter
\def\fps@figure{htbp}
\makeatother
\setlength{\emergencystretch}{3em} % prevent overfull lines
\providecommand{\tightlist}{%
  \setlength{\itemsep}{0pt}\setlength{\parskip}{0pt}}
\setcounter{secnumdepth}{-\maxdimen} % remove section numbering
\usepackage{amsmath}
\usepackage{xcolor}

\title{Finite Population Correction}
\usepackage{etoolbox}
\makeatletter
\providecommand{\subtitle}[1]{% add subtitle to \maketitle
  \apptocmd{\@title}{\par {\large #1 \par}}{}{}
}
\makeatother
\subtitle{Numerical Statistics Fall, 2021}
\author{Jake Underland}
\date{2021-10-01}

\begin{document}
\maketitle

{
\setcounter{tocdepth}{2}
\tableofcontents
}
\hypertarget{lesson}{%
\section{Lesson}\label{lesson}}

Let \(\Pi = \{ x_1, x_2, \dots, x_N\}\) be a finite population of size
\(N\), and suppose we want to estimate an unknown population mean.

\hypertarget{the-population-mean}{%
\subsection{The Population Mean}\label{the-population-mean}}

\[\mu = \frac{1}{N} \sum^N_{i=1} X_i \]

\hypertarget{sample-mean-with-replacement}{%
\subsection{Sample Mean With
Replacement}\label{sample-mean-with-replacement}}

Draw a sample of size \(n\) from the above population
(\(\{X_1, X_2, \dots, X_n\}, \:2 \leq n \leq N\)). Then, the sample mean
is \[\bar{X} = \frac{1}{n}\sum_{i=1}^n X_i\] This is unbiased.
\[E(\bar{X}) = E(\frac{1}{n}\sum_{i=1}^n X_i) = \underbrace{ \frac{1}{n}\sum_{i=1}^n E(X_i)}_{Linearity} =  \frac{1}{n}\sum_{i=1}^n \mu = \mu\]
The above holds because under replacement, because \[\begin{aligned} 
P(\underbrace{X_j}_{sample} = \underbrace{x_i}_{population}) &= \frac{1}{N}\;\; \begin{cases} j = 1, 2, \dots, n \\ i = 1, 2, \dots, N \end{cases} \\
P(X_i = x_k, X_j = x_l) &= \frac{1}{N^2} \dots \color{red}{!!!}\\
\implies E(X_j) &= \sum_{i=1}^{N}x_iP(X_j=x_i) \\
&= \sum_{i=1}^{N}x_i\frac{1}{N} \\
&= \frac{1}{N} \sum^N_{i=1} X_i \\
&= \mu \dots \text{by definition}
\end{aligned}\] However, we know this is unbiased regardless of whether
you \emph{sample with or without replacement}.

\hypertarget{sample-mean-without-replacement}{%
\subsection{Sample Mean Without
Replacement}\label{sample-mean-without-replacement}}

In case of replacement, \[\begin{aligned} 
P(X_j = x_i) &= \frac{1}{N}\;\; \begin{cases} j = 1, 2, \dots, n \\ i = 1, 2, \dots, N \end{cases} \\
P(X_i = x_k, X_j = x_l) &= \begin{cases} 
\frac{1}N(\frac{1}{N-1}) &\dots (k\ne l) \\
0 &\dots (k =l)
\end{cases} \dots \color{red}{!!!}\\
\implies E(X_j) &= \sum_{i=1}^{N}x_iP(X_j=x_i) \\
&= \sum_{i=1}^{N}x_i\frac{1}{N} \\
&= \frac{1}{N} \sum^N_{i=1} X_i \\
&= \mu \dots \text{by definition}
\end{aligned}\] Because while the joint marginal probability changes
from the case with replacement, the marginal probability does not,
meaning the same proof holds for unbiasedness of the sample mean in the
case of no replacement.\\
Why does the marginal probability not change? Think of this case
\(\downarrow\) \[\begin{aligned} 
P(X_3 = a) &= P(X_1 \ne a \cap X_2 \ne a \cap X_3 = a) \\
&= \underbrace{\frac{N-1}{N}}_\text{remove \(a\)} \cdot  \underbrace{\frac{N-2}{N-1}}_\text{remove \(a\) and \(X_1\)} \cdot \underbrace{\frac{1}{N-2}}_\text{remove \(X_1, X_2\)} \\
&= \frac{1}{N}
\end{aligned}\]

\hypertarget{the-population-variance}{%
\subsection{The Population Variance}\label{the-population-variance}}

\[\sigma_2 = \frac{1}{N} \sum_{i=1}^{N} (x_i - \mu)^2\]

\hypertarget{the-sample-variance-with-replacement}{%
\subsection{The Sample Variance With
Replacement}\label{the-sample-variance-with-replacement}}

\[Var(X_j) = \sigma ^2\] Proof:
\[\begin{aligned} Var(X_j) &= E[(X_j - E(X_j))^2] \\
&= E[(X_j - \mu)^2] \\
&= \sum^{N}_{i=1}(x_i - \mu)^2 P(X_j = x_i) \\
&= \sum^{N}_{i=1}(x_i - \mu)^2 \frac{1}{N} \\
&= \frac{1}{N} \sum_{i=1}^{N} (x_i - \mu)^2 \\
&= \sigma^2 \; \Box
\end{aligned}\]

\hypertarget{variance-of-sample-mean-with-replacement}{%
\subsection{Variance of Sample Mean With
Replacement}\label{variance-of-sample-mean-with-replacement}}

\[Var(\bar{X}) = \frac{\sigma^2}{n}\] Proof.

\[\begin{aligned} 
Var(\bar{X}) &= Var(\frac{1}{n}\sum_{i=1}^n X_i) \\
&=\frac{1}{n^2}\sum_{i=1}^nVar(X_i) \dots \color{blue}{Independence} \\
&= \frac{1}{n^2}\sum_{i=1}^n  \sigma^2 \\
&= \frac{\sigma^2}{n} \dots\Box
\end{aligned}\]

\hypertarget{variance-of-sample-mean-without-replacement}{%
\subsection{Variance of Sample Mean Without
Replacement}\label{variance-of-sample-mean-without-replacement}}

First, we start with covariance.
\[Cov(X_i, X_j) = -\frac{\sigma^2}{N-1}\;\;(i\ne j)\] Proof.
\[\begin{aligned}
Cov(X_i, X_j) &= E[(X_i - E[X_i])(X_j - E[X_j])] \\
&= E[(X_i - \mu)(X_j - \mu)] \\
&= \sum_{i=1, j=1}^{N}(X_i - \mu)(X_j - \mu) \\
&= \sum_{k = l, k \ne l}^{N}(x_k - \mu)(x_l - \mu)\underbrace{P(X_i = x_k, X_j = x_l)}_\text{\(=0\) when \(k = l\)} \\
&= \sum_{k \ne l}^{N}(x_k - \mu)(x_l - \mu)\frac{1}{N}(\frac{1}{N-1}) \\
&= \frac{1}{N(N-1)}\sum_{k \ne l}^{N}(x_k - \mu)(x_l - \mu) 
\end{aligned}\] We know that \[\begin{aligned}
\{\sum_{i=1}^N(x_i-\mu)\}^2 &= 0 \\
\implies \sum_{i=1}^{N}(x_i-\mu)^2 + \sum_{k\ne l}(x_k-\mu)(x_l-\mu)&= 0 \\
\implies \sum_{k\ne l}(x_k-\mu)(x_l-\mu) &= -  \sum_{i=1}^{N}(x_i-\mu)^2 \\
&= -N\sigma^2 
\end{aligned}\] Thus, \[\begin{aligned}
Cov(X_i, X_j)
&= \frac{1}{N(N-1)}\sum_{k \ne l}^{N}(x_k - \mu)(x_l - \mu) \\
&= \frac{1}{N(N-1)} (-N\sigma^2) \\
&= -\frac{\sigma^2}{N-1} \dots \Box
\end{aligned}\] Using this, we find that
\[Var(\bar{X}) = \frac{\sigma^2}{n}(\frac{N-n}{N-1})\] Proof.
\[\begin{aligned} 
Var(\bar{X}) &= Var(\frac{1}{n}\sum_{i=1}^n X_i) \\
&=\frac{1}{n^2}Var(\sum_{i=1}^n X_i) \\
&= \frac{1}{n^2}\sum_{i=1}^nVar(X_i) + \frac{1}{n^2}\sum_{i\ne j}Cov(X_i, X_j) \\
&= \frac{\sigma^2}{n} + \frac{n-1}{n}(-\frac{\sigma^2}{N-1})\\
&= \frac{\sigma^2}{n}(1 - \frac{n-1}{N-1})\\
&= \frac{\sigma^2}{n}\underline{(\frac{N-n}{N-1})}_{\dots\text{Finite Population Correction Factor}} \Box
\end{aligned}\]

\hypertarget{notes-on-the-finite-population-correction-factor}{%
\subsection{Notes on the Finite Population Correction
Factor}\label{notes-on-the-finite-population-correction-factor}}

We call the constant \(FPC = \sqrt{\frac{N-n}{N-1}}\) the finite
population correction factor.\\
We can use the FPC to construct confidence intervals for means or
proportions when sampling from a finite population without
replacement.\\
The confidence interval for \(\mu\) is given by:
\[\bar{X} \pm 1.96\frac{\sigma}{\sqrt{n}}\sqrt{\frac{N-n}{N-1}}\] Proof.
\[\begin{aligned}
&\frac{\bar{X} - E(\bar{X})}{\sqrt{Var(\bar{X})}}\sim N(0, 1) \\
\implies &P(-1.96 < \frac{\bar{X} - E(\bar{X})}{\sqrt{Var(\bar{X})}} < 1.96) = 0.95 \\
\implies &P(-1.96 < \frac{\bar{X} - \mu}{\sqrt{\frac{\sigma^2}{n}(\frac{N-n}{N-1})}} < 1.96) = 0.95 \\
\implies &P(\bar{X} - 1.96\frac{\sigma}{\sqrt{n}}\sqrt{\frac{N-n}{N-1}} < \mu < \bar{X} + 1.96\frac{\sigma}{\sqrt{n}}\sqrt{\frac{N-n}{N-1}}) = 0.95 \dots \Box
\end{aligned}\] When sampling is with replacement,
\(Var_{(1)}(\bar{X}) = \frac{\sigma^2}{n}\). When without replacement,
\(Var_{(2)}(\bar{X}) = \frac{\sigma^2}{n}(\frac{N-n}{N-1})\).
\[Var_{(1)}(\bar{X}) = \frac{\sigma^2}{n} > \frac{\sigma^2}{n}(\frac{N-n}{N-1}) = Var_{(2)}(\bar{X})\dots \text{because }(n > 1)\]
From here we see that using the sample mean to estimate \(\mu\) when
drawing from a finite population without replacement has higher accuracy
than doing so with replacement. However, the difference is negligible
the higher the cardinality of population \(N\) and the lower the sample
size \(n\).

\newpage

\hypertarget{exercises}{%
\section{Exercises}\label{exercises}}

\hypertarget{exercise-01}{%
\subsubsection{Exercise 01}\label{exercise-01}}

Solve a linear equation \(\sum_{i=1}^{n}(x_i-c) = 0\) for \(c\), and
verify that its solution is the arithmetic mean \(\bar{x}\).

\textit{Solution.} \[\begin{aligned}
\sum_{i=1}^{n}(x_i-c) &= 0 \\
\implies \sum_{i=1}^{n}x_i &= nc \\
\implies c &= \frac{1}{n}\sum_{i=1}^{n}x_i = \bar{x}
\end{aligned}\]

\end{document}
