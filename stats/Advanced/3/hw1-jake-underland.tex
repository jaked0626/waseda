% Options for packages loaded elsewhere
\PassOptionsToPackage{unicode}{hyperref}
\PassOptionsToPackage{hyphens}{url}
%
\documentclass[
]{article}
\usepackage{lmodern}
\usepackage{amssymb,amsmath}
\usepackage{ifxetex,ifluatex}
\ifnum 0\ifxetex 1\fi\ifluatex 1\fi=0 % if pdftex
  \usepackage[T1]{fontenc}
  \usepackage[utf8]{inputenc}
  \usepackage{textcomp} % provide euro and other symbols
\else % if luatex or xetex
  \usepackage{unicode-math}
  \defaultfontfeatures{Scale=MatchLowercase}
  \defaultfontfeatures[\rmfamily]{Ligatures=TeX,Scale=1}
\fi
% Use upquote if available, for straight quotes in verbatim environments
\IfFileExists{upquote.sty}{\usepackage{upquote}}{}
\IfFileExists{microtype.sty}{% use microtype if available
  \usepackage[]{microtype}
  \UseMicrotypeSet[protrusion]{basicmath} % disable protrusion for tt fonts
}{}
\makeatletter
\@ifundefined{KOMAClassName}{% if non-KOMA class
  \IfFileExists{parskip.sty}{%
    \usepackage{parskip}
  }{% else
    \setlength{\parindent}{0pt}
    \setlength{\parskip}{6pt plus 2pt minus 1pt}}
}{% if KOMA class
  \KOMAoptions{parskip=half}}
\makeatother
\usepackage{xcolor}
\IfFileExists{xurl.sty}{\usepackage{xurl}}{} % add URL line breaks if available
\IfFileExists{bookmark.sty}{\usepackage{bookmark}}{\usepackage{hyperref}}
\hypersetup{
  pdftitle={Homework Assignment 01},
  pdfauthor={Jake Underland},
  hidelinks,
  pdfcreator={LaTeX via pandoc}}
\urlstyle{same} % disable monospaced font for URLs
\usepackage[margin=1in]{geometry}
\usepackage{graphicx,grffile}
\makeatletter
\def\maxwidth{\ifdim\Gin@nat@width>\linewidth\linewidth\else\Gin@nat@width\fi}
\def\maxheight{\ifdim\Gin@nat@height>\textheight\textheight\else\Gin@nat@height\fi}
\makeatother
% Scale images if necessary, so that they will not overflow the page
% margins by default, and it is still possible to overwrite the defaults
% using explicit options in \includegraphics[width, height, ...]{}
\setkeys{Gin}{width=\maxwidth,height=\maxheight,keepaspectratio}
% Set default figure placement to htbp
\makeatletter
\def\fps@figure{htbp}
\makeatother
\setlength{\emergencystretch}{3em} % prevent overfull lines
\providecommand{\tightlist}{%
  \setlength{\itemsep}{0pt}\setlength{\parskip}{0pt}}
\setcounter{secnumdepth}{-\maxdimen} % remove section numbering
\usepackage{amsmath}
\usepackage{xcolor}
\usepackage{bm}

\title{Homework Assignment 01}
\usepackage{etoolbox}
\makeatletter
\providecommand{\subtitle}[1]{% add subtitle to \maketitle
  \apptocmd{\@title}{\par {\large #1 \par}}{}{}
}
\makeatother
\subtitle{Numerical Statistics Fall, 2021}
\author{Jake Underland}
\date{2021-10-15}

\begin{document}
\maketitle

\hypertarget{section}{%
\subsection{1.}\label{section}}

Calculate the arithmetic mean \(\bar{x_5}\) of a data \(4, 6, 8, 9, 13\)
and then express it in the form of an irreducible fraction or an
integer.

\textit{Solution.}\\
\[\begin{aligned}
\bar{x_5} &= \frac{4+ 6+ 8+ 9+ 13}{5} = \frac{40}{5} = 8
\end{aligned}\]

\hypertarget{section-1}{%
\subsection{2.}\label{section-1}}

Suppose we draw a sample of size \(16\) from a finite population of size
\(65\), where sampling is without replacement. Calculate the finite
population correction factor \(FPC\) and then express it in the form of
a decimal to the third decimal places.

\textit{Solution.}\\
\[FPC_{N, n} = \sqrt{\frac{N-n}{N-1}}, \: N = 65, \: n = 16\]
\[FPC_{65,16} = \sqrt{\frac{65-16}{65-1}} = \sqrt{\frac{49}{64}} = \frac{7}{8} = 0.875\]

\hypertarget{section-2}{%
\subsection{3.}\label{section-2}}

Suppose \(X_1, X_2, X_3, X_4, X_5 \stackrel{i.i.d.}{\sim} Ber (0.4)\).
Calculate the probability of the outcome
\(\{X_1 =0, X_2 = 0, X_3 = 1, X_4 = 0, X_5 = 1\}\) and then express it
in the form of a decimal to the fifth decimal places.

\textit{Solution.}\\
\[\begin{aligned}
P(X_1 = 0, X_2 = 0, X_3 = 1, X_4 = 0, X_5 = 1)
&= P(X_1 = 0) P(X_2 = 0) P(X_3 = 1) P(X_4 =0) P(X_5 = 1) \\
&= p^2(1-p)^3 \\
&= (0.4)^2(0.6)^3 \\
&= 0.03456
\end{aligned}\]

\hypertarget{section-3}{%
\subsection{4.}\label{section-3}}

Consider a data
\((x_1, x_2, x_3, x_4) = (-5, a, 1, b) \;(-5 < a < 1 < b)\) whose
arithmetic mean and variance are \(2\) and \(\frac{77}{2}\),
respectively. Find the real numbers \(a\) and \(b\).

\textit{Solution.}\\
From the formula for the arithmetic mean, we have \[\begin{aligned}
\frac{-5 + a + 1 + b}{4} &= 2 \\
-5 + a + 1 + b &= 8 \\
b &= 12 - a\\
\end{aligned}\]

Plug the value of \(b\) into the equation for variance:
\[\begin{aligned}
\frac{(-5 - 2)^2 + (a-2)^2 + (1-2)^2 + (12-a-2)^2}{4} &= \frac{77}{2}\\
49 + a^2 -4a + 4 + 1 + 100 - 20a + a^2 &= 154 \\
2a^2 - 24a &= 0 \\
a(12-a) &= 0 \\
\implies \begin{cases}a = 0 \\ b = 12 \end{cases}, \begin{cases} a = 12 \\ b = 0 \end{cases}&
\end{aligned}\] Since \(-5 < a < 1 < b\), we can rule out the case where
\(a = 12, b = 0\), leaving us with \[a = 0, \;b = 12\dots \Box\]

\hypertarget{section-4}{%
\subsection{5.}\label{section-4}}

Suppose we have a data \(x_1, x_2, \dots , x_n\) of size \(n\). Prove
that the inequality
\[\sum_{i=1}^n(x_i-c)^2 \ge \sum_{i=1}^n(x_i-\bar{x})^2\] holds for any
real number \(c\), where \(\bar{x}\) is the arithmetic mean.

\textit{Solution.}\\
In order to prove
\(\sum_{i=1}^n(x_i-c)^2 \ge \sum_{i=1}^n(x_i-\bar{x})^2\), we first show
\(\sum_{i=1}^n(x_i-c)^2 - \sum_{i=1}^n(x_i-\bar{x})^2 \ge 0\).
\[\begin{aligned}
\sum_{i=1}^n(x_i-c)^2 - \sum_{i=1}^n(x_i-\bar{x})^2 &= \sum_{i=1}^nx_i^2 - \sum_{i=1}^nx_i^2 - 2c\underbrace{\sum_{i=1}^nx_i}_{n\bar{x}} + 2\bar{x} \underbrace{\sum_{i=1}^nx_i}_{n\bar{x}} + nc^2 -n\bar{x}^2\\
&= -2cn\bar{x} + 2n\bar{x}^2 + nc^2 - n \bar{x}^2 \\
&= n(\bar{x}^2 -2c\bar{x}+ c^2) \\
&= n(\bar{x} - c)^2 \ge 0
\end{aligned}\] Where the last inequality holds because \(n \ge 0\) by
definition and \((\bar{x} - c)^2\) is squared and thus at least \(0\).
Then, we have proven that

\[\sum_{i=1}^n(x_i-c)^2 - \sum_{i=1}^n(x_i-\bar{x})^2 \ge 0\] From which
follows:
\[\sum_{i=1}^n(x_i-c)^2 \ge \sum_{i=1}^n(x_i-\bar{x})^2 \dots \Box\]

\end{document}
