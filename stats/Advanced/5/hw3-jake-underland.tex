% Options for packages loaded elsewhere
\PassOptionsToPackage{unicode}{hyperref}
\PassOptionsToPackage{hyphens}{url}
%
\documentclass[
]{article}
\usepackage{lmodern}
\usepackage{amssymb,amsmath}
\usepackage{ifxetex,ifluatex}
\ifnum 0\ifxetex 1\fi\ifluatex 1\fi=0 % if pdftex
  \usepackage[T1]{fontenc}
  \usepackage[utf8]{inputenc}
  \usepackage{textcomp} % provide euro and other symbols
\else % if luatex or xetex
  \usepackage{unicode-math}
  \defaultfontfeatures{Scale=MatchLowercase}
  \defaultfontfeatures[\rmfamily]{Ligatures=TeX,Scale=1}
\fi
% Use upquote if available, for straight quotes in verbatim environments
\IfFileExists{upquote.sty}{\usepackage{upquote}}{}
\IfFileExists{microtype.sty}{% use microtype if available
  \usepackage[]{microtype}
  \UseMicrotypeSet[protrusion]{basicmath} % disable protrusion for tt fonts
}{}
\makeatletter
\@ifundefined{KOMAClassName}{% if non-KOMA class
  \IfFileExists{parskip.sty}{%
    \usepackage{parskip}
  }{% else
    \setlength{\parindent}{0pt}
    \setlength{\parskip}{6pt plus 2pt minus 1pt}}
}{% if KOMA class
  \KOMAoptions{parskip=half}}
\makeatother
\usepackage{xcolor}
\IfFileExists{xurl.sty}{\usepackage{xurl}}{} % add URL line breaks if available
\IfFileExists{bookmark.sty}{\usepackage{bookmark}}{\usepackage{hyperref}}
\hypersetup{
  pdftitle={Homework Assignment 02},
  pdfauthor={Jake Underland - 1A193008},
  hidelinks,
  pdfcreator={LaTeX via pandoc}}
\urlstyle{same} % disable monospaced font for URLs
\usepackage[margin=1in]{geometry}
\usepackage{color}
\usepackage{fancyvrb}
\newcommand{\VerbBar}{|}
\newcommand{\VERB}{\Verb[commandchars=\\\{\}]}
\DefineVerbatimEnvironment{Highlighting}{Verbatim}{commandchars=\\\{\}}
% Add ',fontsize=\small' for more characters per line
\usepackage{framed}
\definecolor{shadecolor}{RGB}{248,248,248}
\newenvironment{Shaded}{\begin{snugshade}}{\end{snugshade}}
\newcommand{\AlertTok}[1]{\textcolor[rgb]{0.94,0.16,0.16}{#1}}
\newcommand{\AnnotationTok}[1]{\textcolor[rgb]{0.56,0.35,0.01}{\textbf{\textit{#1}}}}
\newcommand{\AttributeTok}[1]{\textcolor[rgb]{0.77,0.63,0.00}{#1}}
\newcommand{\BaseNTok}[1]{\textcolor[rgb]{0.00,0.00,0.81}{#1}}
\newcommand{\BuiltInTok}[1]{#1}
\newcommand{\CharTok}[1]{\textcolor[rgb]{0.31,0.60,0.02}{#1}}
\newcommand{\CommentTok}[1]{\textcolor[rgb]{0.56,0.35,0.01}{\textit{#1}}}
\newcommand{\CommentVarTok}[1]{\textcolor[rgb]{0.56,0.35,0.01}{\textbf{\textit{#1}}}}
\newcommand{\ConstantTok}[1]{\textcolor[rgb]{0.00,0.00,0.00}{#1}}
\newcommand{\ControlFlowTok}[1]{\textcolor[rgb]{0.13,0.29,0.53}{\textbf{#1}}}
\newcommand{\DataTypeTok}[1]{\textcolor[rgb]{0.13,0.29,0.53}{#1}}
\newcommand{\DecValTok}[1]{\textcolor[rgb]{0.00,0.00,0.81}{#1}}
\newcommand{\DocumentationTok}[1]{\textcolor[rgb]{0.56,0.35,0.01}{\textbf{\textit{#1}}}}
\newcommand{\ErrorTok}[1]{\textcolor[rgb]{0.64,0.00,0.00}{\textbf{#1}}}
\newcommand{\ExtensionTok}[1]{#1}
\newcommand{\FloatTok}[1]{\textcolor[rgb]{0.00,0.00,0.81}{#1}}
\newcommand{\FunctionTok}[1]{\textcolor[rgb]{0.00,0.00,0.00}{#1}}
\newcommand{\ImportTok}[1]{#1}
\newcommand{\InformationTok}[1]{\textcolor[rgb]{0.56,0.35,0.01}{\textbf{\textit{#1}}}}
\newcommand{\KeywordTok}[1]{\textcolor[rgb]{0.13,0.29,0.53}{\textbf{#1}}}
\newcommand{\NormalTok}[1]{#1}
\newcommand{\OperatorTok}[1]{\textcolor[rgb]{0.81,0.36,0.00}{\textbf{#1}}}
\newcommand{\OtherTok}[1]{\textcolor[rgb]{0.56,0.35,0.01}{#1}}
\newcommand{\PreprocessorTok}[1]{\textcolor[rgb]{0.56,0.35,0.01}{\textit{#1}}}
\newcommand{\RegionMarkerTok}[1]{#1}
\newcommand{\SpecialCharTok}[1]{\textcolor[rgb]{0.00,0.00,0.00}{#1}}
\newcommand{\SpecialStringTok}[1]{\textcolor[rgb]{0.31,0.60,0.02}{#1}}
\newcommand{\StringTok}[1]{\textcolor[rgb]{0.31,0.60,0.02}{#1}}
\newcommand{\VariableTok}[1]{\textcolor[rgb]{0.00,0.00,0.00}{#1}}
\newcommand{\VerbatimStringTok}[1]{\textcolor[rgb]{0.31,0.60,0.02}{#1}}
\newcommand{\WarningTok}[1]{\textcolor[rgb]{0.56,0.35,0.01}{\textbf{\textit{#1}}}}
\usepackage{graphicx,grffile}
\makeatletter
\def\maxwidth{\ifdim\Gin@nat@width>\linewidth\linewidth\else\Gin@nat@width\fi}
\def\maxheight{\ifdim\Gin@nat@height>\textheight\textheight\else\Gin@nat@height\fi}
\makeatother
% Scale images if necessary, so that they will not overflow the page
% margins by default, and it is still possible to overwrite the defaults
% using explicit options in \includegraphics[width, height, ...]{}
\setkeys{Gin}{width=\maxwidth,height=\maxheight,keepaspectratio}
% Set default figure placement to htbp
\makeatletter
\def\fps@figure{htbp}
\makeatother
\setlength{\emergencystretch}{3em} % prevent overfull lines
\providecommand{\tightlist}{%
  \setlength{\itemsep}{0pt}\setlength{\parskip}{0pt}}
\setcounter{secnumdepth}{-\maxdimen} % remove section numbering
\usepackage{amsmath}
\usepackage{xcolor}
\usepackage{bm}

\title{Homework Assignment 02}
\usepackage{etoolbox}
\makeatletter
\providecommand{\subtitle}[1]{% add subtitle to \maketitle
  \apptocmd{\@title}{\par {\large #1 \par}}{}{}
}
\makeatother
\subtitle{Numerical Statistics Fall, 2022}
\author{Jake Underland - 1A193008}
\date{2022-01-24}

\begin{document}
\maketitle

\hypertarget{section}{%
\section{1.}\label{section}}

Suppose that five people, \(a, b, c, d\) and \(e\), entered their photos
in an exhibition, and judges \(X\) and \(Y\) scored the photos within
the range of \(0\) to \(50\) points and \(0\) to \(100\) points,
respectively:\\

\begin{center}
\begin{tabular}{|c|c|c|c|c|c|}
\hline
 & a & b & c & d & e\\
\hline
X & 32 & 18 & 39 & 47 & 26\\
\hline
Y & 90 & 65 & 45 & 80 & 70 \\
\hline
\end{tabular}
\end{center}

In this case calculate the Spearman rank correlation coefficient
\(r_s\), and then express it as an irreducible fraction.

\textbf{Solution}:\\
The table converted to express only the ordinal evaluations of \(X\) and
\(Y\) is as below:

\begin{center}
\begin{tabular}{|c|c|c|c|c|c|}
\hline
 & a & b & c & d & e\\
\hline
X & 3 & 1 & 4 & 5 & 2\\
\hline
Y & 5 & 2 & 1 & 4 & 3 \\
\hline
\end{tabular}
\end{center}

The formula for Spearman rank correlation coefficient is
\[r_s = 1 - \frac{6}{n^3-n} \sum^n_{i=1}(R_i - R_i')^2\] Thus, plugging
the ranked data in we get \[
\begin{aligned}
r_s &= 1 - \frac{6}{5^3-5} \left\{(3-5)^2 + (1 - 2)^2 + (4-1)^2 + (5-4)^2 + (2-3)^2\right\} \\
&= \frac{24}{120} = \frac{1}{5}
\end{aligned}
\] Code for calculating the Spearman rank correlation coefficient is
below.

\begin{Shaded}
\begin{Highlighting}[]
\ImportTok{import}\NormalTok{ numpy }\ImportTok{as}\NormalTok{ np}
\NormalTok{X }\OperatorTok{=}\NormalTok{ [}\DecValTok{32}\NormalTok{, }\DecValTok{18}\NormalTok{, }\DecValTok{39}\NormalTok{, }\DecValTok{47}\NormalTok{, }\DecValTok{26}\NormalTok{]}
\NormalTok{Y }\OperatorTok{=}\NormalTok{ [}\DecValTok{90}\NormalTok{, }\DecValTok{65}\NormalTok{, }\DecValTok{45}\NormalTok{, }\DecValTok{80}\NormalTok{, }\DecValTok{70}\NormalTok{]}

\KeywordTok{def}\NormalTok{ rank_vector(vec): }
    \ControlFlowTok{return}\NormalTok{ [(}\BuiltInTok{sorted}\NormalTok{(vec).index(x) }\OperatorTok{+} \DecValTok{1}\NormalTok{) }\ControlFlowTok{for}\NormalTok{ x }\KeywordTok{in}\NormalTok{ vec]}
    
\KeywordTok{def}\NormalTok{ compute_spearman(X, Y):}
    \CommentTok{'''}
\CommentTok{    inputs: two vectors}
\CommentTok{    outputs: spearman coefficient of two vectors}
\CommentTok{    '''}
    \ControlFlowTok{assert} \BuiltInTok{len}\NormalTok{(X) }\OperatorTok{==} \BuiltInTok{len}\NormalTok{(Y)}
\NormalTok{    n }\OperatorTok{=} \BuiltInTok{len}\NormalTok{(X)}
\NormalTok{    data }\OperatorTok{=}\NormalTok{ np.array([rank_vector(X), rank_vector(Y)])}
\NormalTok{    rs }\OperatorTok{=} \DecValTok{1} \OperatorTok{-}\NormalTok{ (}\DecValTok{6} \OperatorTok{/}\NormalTok{ (n}\OperatorTok{**}\DecValTok{3} \OperatorTok{-}\NormalTok{ n)) }\OperatorTok{*} \BuiltInTok{sum}\NormalTok{((data[}\DecValTok{0}\NormalTok{,] }\OperatorTok{-}\NormalTok{ data[}\DecValTok{1}\NormalTok{,])}\OperatorTok{**}\DecValTok{2}\NormalTok{)}
    \ControlFlowTok{return}\NormalTok{ rs}
\end{Highlighting}
\end{Shaded}

\hypertarget{section-1}{%
\section{2.}\label{section-1}}

Consider the following data:

\begin{center}
\begin{tabular}{|c|c|c|c|c|c|c|}
\hline
x & -1 & 0 & 1 & 3 & 4 & 5\\
\hline
y & 8 & 6 & 5 & 2 & 0 & -3\\
\hline
\end{tabular}
\end{center}

\begin{enumerate}
\item[(2-1)] Find the linear regression line, $y = \alpha + \beta x$, of $y$ on $x$ based on the data, and then express $\alpha$ and $\beta$ as irreducible fractions.
\end{enumerate}

\textbf{Solution.}

In the following simple regression model
\(y = \beta_0 + \beta_1 x + e_i\), the best linear predictors for the
two coefficients under squared loss
are\((\hat{\beta_0}, \hat{\beta_1}) \in \arg \! \min_{\beta_0, \beta_1} \sum_i(y_i - \beta_0 - \beta_1 x_i)^2\).
We can compute these as, \[ \begin{aligned}
\hat{\beta_0} &= \bar{y} - \hat{\beta_1}\bar{x}\\
\hat{\beta_1} &= \frac{\sum_{i=1}(y_i - \bar{y})(x_i - \bar{x})}{\sum_{i=1}(x_i  - \bar{x})^2 }
\end{aligned}\] Then, following the above formula, \[\begin{aligned}
\bar{x} &= 2\\
\bar{y} &= 3\\
\beta &= \frac{(-1 - 2)(8-3) + (0 - 2)(6 - 3) + (1 - 2)(5 - 3) + (3-2)(2-3) + (4 - 2)(0 - 3) + (5 - 2)(-3 - 3)}{(-1 - 2)^2 + (0 - 2)^2 + (1 - 2)^2 + (3-2)^2 + (4 - 2)^2 + (5 - 2)^2} \\
&= \frac{-12}{7} \\
\alpha &= 3 - \frac{-12}{7} \times 2 = \frac{45}{7} \dots \Box
\end{aligned}\] Code can be found below:

\begin{Shaded}
\begin{Highlighting}[]
\KeywordTok{def}\NormalTok{ estimate_beta1(x, y):}
\NormalTok{    x }\OperatorTok{=}\NormalTok{ np.array(x)}
\NormalTok{    y }\OperatorTok{=}\NormalTok{ np.array(y)}
\NormalTok{    numerator }\OperatorTok{=}\NormalTok{ np.}\BuiltInTok{sum}\NormalTok{((x }\OperatorTok{-}\NormalTok{ np.mean(x)) }\OperatorTok{*}\NormalTok{ (y }\OperatorTok{-}\NormalTok{ np.mean(y)))}
\NormalTok{    denominator }\OperatorTok{=}\NormalTok{ np.}\BuiltInTok{sum}\NormalTok{((x }\OperatorTok{-}\NormalTok{ np.mean(x)) }\OperatorTok{**} \DecValTok{2}\NormalTok{)}
    \ControlFlowTok{return}\NormalTok{ reduce_frac(numerator, denominator)}

\KeywordTok{def}\NormalTok{ reduce_frac(num, denom):}
\NormalTok{    x }\OperatorTok{=}\NormalTok{ gcd(num, denom)}
    \ControlFlowTok{return} \StringTok{"}\SpecialCharTok{\{\}}\StringTok{/}\SpecialCharTok{\{\}}\StringTok{"}\NormalTok{.}\BuiltInTok{format}\NormalTok{(num}\OperatorTok{/}\NormalTok{x, denom}\OperatorTok{/}\NormalTok{x)}
    
\KeywordTok{def}\NormalTok{ gcd(m, n):}
\NormalTok{    r }\OperatorTok{=}\NormalTok{ m }\OperatorTok{%}\NormalTok{ n}
    \ControlFlowTok{return}\NormalTok{ n }\ControlFlowTok{if} \KeywordTok{not}\NormalTok{ r }\ControlFlowTok{else}\NormalTok{ gcd(n, r)}
\end{Highlighting}
\end{Shaded}

\begin{Shaded}
\begin{Highlighting}[]
\NormalTok{x }\OperatorTok{=}\NormalTok{ [}\OperatorTok{-}\DecValTok{1}\NormalTok{, }\DecValTok{0}\NormalTok{, }\DecValTok{1}\NormalTok{, }\DecValTok{3}\NormalTok{, }\DecValTok{4}\NormalTok{, }\DecValTok{5}\NormalTok{]}
\NormalTok{y }\OperatorTok{=}\NormalTok{ [}\DecValTok{8}\NormalTok{, }\DecValTok{6}\NormalTok{, }\DecValTok{5}\NormalTok{, }\DecValTok{2}\NormalTok{, }\DecValTok{0}\NormalTok{, }\DecValTok{-3}\NormalTok{]}
\BuiltInTok{print}\NormalTok{(estimate_beta1(x, y))}
\end{Highlighting}
\end{Shaded}

\begin{verbatim}
## -12.0/7.0
\end{verbatim}

\begin{enumerate}
\item[(2-2)] Find the linear regression line, $x = \gamma + \delta y$ of $x$ on $y$ based on the data, and then express $\gamma$ and $\delta$ as irreducible fractions.
\end{enumerate}

\textbf{Solution.}\\
Similarly,

\begin{Shaded}
\begin{Highlighting}[]
\BuiltInTok{print}\NormalTok{(estimate_beta1(y, x))}
\end{Highlighting}
\end{Shaded}

\begin{verbatim}
## -4.0/7.0
\end{verbatim}

Thus, \[\begin{aligned}
\delta &= -\frac{4}{7} \\
\gamma &= 2 - \frac{-4}{7} \times 3 = \frac{26}{7}
\end{aligned}\]

\end{document}
