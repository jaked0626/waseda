% Options for packages loaded elsewhere
\PassOptionsToPackage{unicode}{hyperref}
\PassOptionsToPackage{hyphens}{url}
%
\documentclass[
]{article}
\usepackage{lmodern}
\usepackage{amssymb,amsmath}
\usepackage{ifxetex,ifluatex}
\ifnum 0\ifxetex 1\fi\ifluatex 1\fi=0 % if pdftex
  \usepackage[T1]{fontenc}
  \usepackage[utf8]{inputenc}
  \usepackage{textcomp} % provide euro and other symbols
\else % if luatex or xetex
  \usepackage{unicode-math}
  \defaultfontfeatures{Scale=MatchLowercase}
  \defaultfontfeatures[\rmfamily]{Ligatures=TeX,Scale=1}
\fi
% Use upquote if available, for straight quotes in verbatim environments
\IfFileExists{upquote.sty}{\usepackage{upquote}}{}
\IfFileExists{microtype.sty}{% use microtype if available
  \usepackage[]{microtype}
  \UseMicrotypeSet[protrusion]{basicmath} % disable protrusion for tt fonts
}{}
\makeatletter
\@ifundefined{KOMAClassName}{% if non-KOMA class
  \IfFileExists{parskip.sty}{%
    \usepackage{parskip}
  }{% else
    \setlength{\parindent}{0pt}
    \setlength{\parskip}{6pt plus 2pt minus 1pt}}
}{% if KOMA class
  \KOMAoptions{parskip=half}}
\makeatother
\usepackage{xcolor}
\IfFileExists{xurl.sty}{\usepackage{xurl}}{} % add URL line breaks if available
\IfFileExists{bookmark.sty}{\usepackage{bookmark}}{\usepackage{hyperref}}
\hypersetup{
  pdftitle={Homework Assignment 02},
  pdfauthor={Jake Underland - 1A193008},
  hidelinks,
  pdfcreator={LaTeX via pandoc}}
\urlstyle{same} % disable monospaced font for URLs
\usepackage[margin=1in]{geometry}
\usepackage{color}
\usepackage{fancyvrb}
\newcommand{\VerbBar}{|}
\newcommand{\VERB}{\Verb[commandchars=\\\{\}]}
\DefineVerbatimEnvironment{Highlighting}{Verbatim}{commandchars=\\\{\}}
% Add ',fontsize=\small' for more characters per line
\usepackage{framed}
\definecolor{shadecolor}{RGB}{248,248,248}
\newenvironment{Shaded}{\begin{snugshade}}{\end{snugshade}}
\newcommand{\AlertTok}[1]{\textcolor[rgb]{0.94,0.16,0.16}{#1}}
\newcommand{\AnnotationTok}[1]{\textcolor[rgb]{0.56,0.35,0.01}{\textbf{\textit{#1}}}}
\newcommand{\AttributeTok}[1]{\textcolor[rgb]{0.77,0.63,0.00}{#1}}
\newcommand{\BaseNTok}[1]{\textcolor[rgb]{0.00,0.00,0.81}{#1}}
\newcommand{\BuiltInTok}[1]{#1}
\newcommand{\CharTok}[1]{\textcolor[rgb]{0.31,0.60,0.02}{#1}}
\newcommand{\CommentTok}[1]{\textcolor[rgb]{0.56,0.35,0.01}{\textit{#1}}}
\newcommand{\CommentVarTok}[1]{\textcolor[rgb]{0.56,0.35,0.01}{\textbf{\textit{#1}}}}
\newcommand{\ConstantTok}[1]{\textcolor[rgb]{0.00,0.00,0.00}{#1}}
\newcommand{\ControlFlowTok}[1]{\textcolor[rgb]{0.13,0.29,0.53}{\textbf{#1}}}
\newcommand{\DataTypeTok}[1]{\textcolor[rgb]{0.13,0.29,0.53}{#1}}
\newcommand{\DecValTok}[1]{\textcolor[rgb]{0.00,0.00,0.81}{#1}}
\newcommand{\DocumentationTok}[1]{\textcolor[rgb]{0.56,0.35,0.01}{\textbf{\textit{#1}}}}
\newcommand{\ErrorTok}[1]{\textcolor[rgb]{0.64,0.00,0.00}{\textbf{#1}}}
\newcommand{\ExtensionTok}[1]{#1}
\newcommand{\FloatTok}[1]{\textcolor[rgb]{0.00,0.00,0.81}{#1}}
\newcommand{\FunctionTok}[1]{\textcolor[rgb]{0.00,0.00,0.00}{#1}}
\newcommand{\ImportTok}[1]{#1}
\newcommand{\InformationTok}[1]{\textcolor[rgb]{0.56,0.35,0.01}{\textbf{\textit{#1}}}}
\newcommand{\KeywordTok}[1]{\textcolor[rgb]{0.13,0.29,0.53}{\textbf{#1}}}
\newcommand{\NormalTok}[1]{#1}
\newcommand{\OperatorTok}[1]{\textcolor[rgb]{0.81,0.36,0.00}{\textbf{#1}}}
\newcommand{\OtherTok}[1]{\textcolor[rgb]{0.56,0.35,0.01}{#1}}
\newcommand{\PreprocessorTok}[1]{\textcolor[rgb]{0.56,0.35,0.01}{\textit{#1}}}
\newcommand{\RegionMarkerTok}[1]{#1}
\newcommand{\SpecialCharTok}[1]{\textcolor[rgb]{0.00,0.00,0.00}{#1}}
\newcommand{\SpecialStringTok}[1]{\textcolor[rgb]{0.31,0.60,0.02}{#1}}
\newcommand{\StringTok}[1]{\textcolor[rgb]{0.31,0.60,0.02}{#1}}
\newcommand{\VariableTok}[1]{\textcolor[rgb]{0.00,0.00,0.00}{#1}}
\newcommand{\VerbatimStringTok}[1]{\textcolor[rgb]{0.31,0.60,0.02}{#1}}
\newcommand{\WarningTok}[1]{\textcolor[rgb]{0.56,0.35,0.01}{\textbf{\textit{#1}}}}
\usepackage{graphicx,grffile}
\makeatletter
\def\maxwidth{\ifdim\Gin@nat@width>\linewidth\linewidth\else\Gin@nat@width\fi}
\def\maxheight{\ifdim\Gin@nat@height>\textheight\textheight\else\Gin@nat@height\fi}
\makeatother
% Scale images if necessary, so that they will not overflow the page
% margins by default, and it is still possible to overwrite the defaults
% using explicit options in \includegraphics[width, height, ...]{}
\setkeys{Gin}{width=\maxwidth,height=\maxheight,keepaspectratio}
% Set default figure placement to htbp
\makeatletter
\def\fps@figure{htbp}
\makeatother
\setlength{\emergencystretch}{3em} % prevent overfull lines
\providecommand{\tightlist}{%
  \setlength{\itemsep}{0pt}\setlength{\parskip}{0pt}}
\setcounter{secnumdepth}{-\maxdimen} % remove section numbering
\usepackage{amsmath}
\usepackage{xcolor}
\usepackage{bm}

\title{Homework Assignment 02}
\usepackage{etoolbox}
\makeatletter
\providecommand{\subtitle}[1]{% add subtitle to \maketitle
  \apptocmd{\@title}{\par {\large #1 \par}}{}{}
}
\makeatother
\subtitle{Numerical Statistics Fall, 2022}
\author{Jake Underland - 1A193008}
\date{2022-01-04}

\begin{document}
\maketitle

\hypertarget{section}{%
\section{1.}\label{section}}

Consider a stock that grows by 12\% in year one, declines by 8\% in year
two, declines by 3\% in year three and grows by 42\% in year four. In
this case calculate the average growth rate of the stock, and then
express it in the form of a decimal to the second decimal places. (When
calculating the fourth root of a number using a calculator, you need
only apply the square root twice.) (2 points)

\textbf{Solution}:

\[\sqrt[4]{(1 + 0.12) \times (1 - 0.08) \times (1 - 0.03) \times (1 + 0.42)} \approx 1.09\dots \Box\]

\begin{Shaded}
\begin{Highlighting}[]
\KeywordTok{sqrt}\NormalTok{(}\KeywordTok{sqrt}\NormalTok{(}\FloatTok{1.12} \OperatorTok{*}\StringTok{ }\NormalTok{(}\DecValTok{1} \OperatorTok{-}\StringTok{ }\FloatTok{0.08}\NormalTok{) }\OperatorTok{*}\StringTok{ }\NormalTok{(}\DecValTok{1} \OperatorTok{-}\StringTok{ }\FloatTok{0.03}\NormalTok{) }\OperatorTok{*}\StringTok{ }\NormalTok{(}\DecValTok{1} \OperatorTok{+}\StringTok{ }\FloatTok{0.42}\NormalTok{)))}
\end{Highlighting}
\end{Shaded}

\begin{verbatim}
## [1] 1.091482
\end{verbatim}

\hypertarget{section-1}{%
\section{2.}\label{section-1}}

Suppose we want to make a histogram based on a data
\(x_1, x_2, \dots, x_{128}\) using the Sturges rule, where we expect the
histogram to be symmetrical. In this case find the number of classes,
\(k\), in the corresponding frequency distribution. (2 points)

\textbf{Solution}:

\[ k \approx 1 + \log_2 128 = 8\dots \Box\]

\begin{Shaded}
\begin{Highlighting}[]
\DecValTok{1} \OperatorTok{+}\StringTok{ }\KeywordTok{log}\NormalTok{(}\DecValTok{128}\NormalTok{, }\DecValTok{2}\NormalTok{)}
\end{Highlighting}
\end{Shaded}

\begin{verbatim}
## [1] 8
\end{verbatim}

\hypertarget{section-2}{%
\section{3.}\label{section-2}}

Consider the linear regression of \(y\) on \(x\) based on a data:

\begin{tabular}{|c|c|c|c|c|c|}
\hline 
$x$ & 0 & 1 & 2 & 3 & 4 \\
\hline 
$y$ & 2 & 1 & 4 & 3 & 5 \\
\hline
\end{tabular}

Using the Pearson's correlation coefficient \(r_{xy}\), calculate the
coefficient of determination \(R^2\) of the regression, and then express
it as an irreducible fraction. (7 points)

\textbf{Solution}:

\begin{align*} \bar{x} &= \frac{(0 + 1 + 2 + 3 + 4)}5  =2\\
\bar{y} &= \frac{(2 + 1 + 4 + 3 + 5)}5 = 3
\end{align*}

\[\begin{array}{ccccccc}
x & y & x - \bar{x} & y - \bar{y} & (x - \bar{x})(y - \bar{y}) & (x - \bar{x})^2 & (y - \bar{y})^2 \\
\hline 
0 & 2 & -2 & -1 & 2 & 4 & 1 \\
1 & 1 & -1 & -2 & 2 & 1 & 4 \\
2 & 4 & 0  &  1 & 0 & 0 & 1 \\
3 & 3 & 1  &  0 & 0 & 1 & 0 \\
4 & 5 & 2  &  2 & 4 & 4 & 4 \\
\hline 
\text{Total} && 0 & 0 & 8 & 10 & 10
\end{array}\]

\hfill\(\underline{\implies r_{xy} = \frac{8}{\sqrt{10\times 10}} = \frac{4}{5}} \ldots\Box\)

\hypertarget{section-3}{%
\section{4.}\label{section-3}}

Suppose that six people, a, b, c, d, e and f, entered their paintings in
an exhibition, and judges X and Y scored the paintings within the range
of 0 to 50 points and 0 to 100 points, respectively:

\begin{tabular}{|c|c|c|c|c|c|c|}
\hline
&a&b&c&d&e&f\\
\hline
X&50&45&25&30&35&10\\
\hline
Y&96&75&88&54&29&42 \\
\hline
\end{tabular}

In this case, calculate the Kendall rank correlation coefficient
\(r_k\), and then express it as an irreducible fraction. (9 points)

\textbf{Solution}:

The above data points ranked are as below:

\begin{tabular}{|c|c|c|c|c|c|c|}
\hline
&a&b&c&d&e&f\\
\hline
X&6&5&2&3&4&1\\
\hline
Y&6&4&5&3&1&2 \\
\hline
\end{tabular}

Since there are 6 data points, the sum of the number of concordant pairs
(\(C\)) and discordant pairs (\(D\)) is
\(C + D = {6\choose 2} = \frac{6\cdot 5}{2} = 15\). I count the number
of concordant and discordant pairs below, using the given definition:

\begin{Shaded}
\begin{Highlighting}[]
\KeywordTok{def}\NormalTok{ countCD(data):}
\NormalTok{  C }\OperatorTok{=}\NormalTok{ D }\OperatorTok{=} \DecValTok{0}
  \ControlFlowTok{for}\NormalTok{ i, tup1 }\KeywordTok{in} \BuiltInTok{enumerate}\NormalTok{(data[:}\OperatorTok{-}\DecValTok{1}\NormalTok{]):}
    \ControlFlowTok{for}\NormalTok{ tup2 }\KeywordTok{in}\NormalTok{ data[i}\OperatorTok{+}\DecValTok{1}\NormalTok{:]:}
      \ControlFlowTok{if}\NormalTok{ (tup1[}\DecValTok{0}\NormalTok{] }\OperatorTok{>}\NormalTok{ tup2[}\DecValTok{0}\NormalTok{] }\KeywordTok{and}\NormalTok{ tup1[}\DecValTok{1}\NormalTok{] }\OperatorTok{>}\NormalTok{ tup2[}\DecValTok{1}\NormalTok{]) }\OperatorTok{\textbackslash{}}
      \KeywordTok{or}\NormalTok{ (tup1[}\DecValTok{0}\NormalTok{] }\OperatorTok{<}\NormalTok{ tup2[}\DecValTok{0}\NormalTok{] }\KeywordTok{and}\NormalTok{ tup1[}\DecValTok{1}\NormalTok{] }\OperatorTok{<}\NormalTok{ tup2[}\DecValTok{1}\NormalTok{]):}
\NormalTok{        C }\OperatorTok{+=} \DecValTok{1}
      \ControlFlowTok{else}\NormalTok{:}
\NormalTok{        D }\OperatorTok{+=} \DecValTok{1}
  \ControlFlowTok{return}\NormalTok{ C, D}
\end{Highlighting}
\end{Shaded}

\newpage

\begin{Shaded}
\begin{Highlighting}[]
\NormalTok{data }\OperatorTok{=}\NormalTok{ [(}\DecValTok{6}\NormalTok{, }\DecValTok{6}\NormalTok{), (}\DecValTok{5}\NormalTok{, }\DecValTok{4}\NormalTok{), (}\DecValTok{2}\NormalTok{, }\DecValTok{5}\NormalTok{), (}\DecValTok{3}\NormalTok{, }\DecValTok{3}\NormalTok{), (}\DecValTok{4}\NormalTok{, }\DecValTok{1}\NormalTok{), (}\DecValTok{1}\NormalTok{, }\DecValTok{2}\NormalTok{)]}

\BuiltInTok{print} \StringTok{"Number of concordant pairs is:"}\NormalTok{, countCD(data)[}\DecValTok{0}\NormalTok{], }\OperatorTok{\textbackslash{}}
\CommentTok{"}\CharTok{\textbackslash{}n}\CommentTok{Number of discordant pairs is:"}\NormalTok{, countCD(data)[}\DecValTok{1}\NormalTok{]}
\end{Highlighting}
\end{Shaded}

\begin{verbatim}
## Number of concordant pairs is: 10 
## Number of discordant pairs is: 5
\end{verbatim}

Thus, using the formula for the Kendall rank correlation coefficient,
\[r_k = \frac{C - D}{C + D} = \frac{10 - 5}{15} = \frac{1}{3}\dots \Box\]

\end{document}
